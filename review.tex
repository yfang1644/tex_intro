%%============================================================================
%%
%%       Filename:  review.tex
%%
%%    Description:  
%%
%%        Version:  1.0
%%        Created:  09/23/2018 04:31:01 PM
%%       Revision:  none
%%
%%         Author:  Fang Yuan (yfang@nju.edu.cn)
%%   Organization:  nju
%%      Copyright:  Copyright (c) 2018, Fang Yuan
%%
%%          Notes:  
%%                
%%============================================================================
\documentclass[a4paper]{article}
%不同环境,中文字体配置可能会有不同
\usepackage[fontset=ubuntu]{ctex}
%右边多留出1cm,作为边注
\usepackage[left=3cm,right=4cm]{geometry}

%下面是更好的参考文献格式。但为了避免缺少数据库导致文字破碎,此处不用
%\usepackage{natbib}
%\setcitestyle{authoryear}

\title{网络暴力}

\author{Tyler Maxwell}

\begin{document}
\maketitle

\begin{abstract}
本研究目的是探索影响人们成为网络暴力的不同因素。
目的是分析出统计特征,是否物理影响到这种上升趋势。
\end{abstract}

关键词:~~网络暴力~~受害人~~社交媒体
\footnote{原文在https://libguides.uwf.edu/c.php?g=215199\&p=1420828。
作者Tyler Maxwell, University of West Florida,
综述写作方法参考\cite{rand2009}。
非本人专业,不保证信息的正确性。英语水平有限,胡言乱语在所难免。
}


\section{引言}
\marginpar{引言部分,介绍研究背景}
人们一直不得不面对欺凌现象。随着技术和社交媒体的发展,一些青春期少年和
成年人一样,无法逃避来自学校和工作场所的骚扰。这一社会现象
被人们认识到是网络暴力 (以下简称网暴)。
Willard (2004)\cite{Willard2004}列举了8种不同的网暴形式,
包括热战、骚扰、诽谤、假冒、泄露隐私、欺诈、排挤和网络跟踪。

在过去的几年中,新闻报道过几个事例,有人因被同伴们持续的嘲笑而轻生。
基于此,研究人员开始关注这一新的社会现象,开展了大量研究,分析网暴
的各个方面,如施暴者类型、网暴普遍性,以及对受害者的影响。
但是未被研究的是,是否缺乏人身威胁影响到人们可能变成网络暴力。

本研究延伸到网暴的动机,网暴对施暴者和受害人的影响,以及二者的关系。
我们还将考察网暴的不同技术面,以及受害人克服骚扰的技术。

\section{研究现状}
而今现在眼目下,网暴成为一种新的社会现象。
\marginpar{Thematic--按主题方式组织材料}
它常常让学生处于孤独、无助的状态。Faucher, Jackson 和
Cassidy(2014)\cite{faucher2014}调查了加拿大4所大学的 1925 名学生,
发现 24.1\% 的学生在过去的一年中受到过网暴的伤害。这一惊人的数字表明,
几乎每四个人中就有一个是网暴的受害者。然而,这一统计数字有趣在于,
当研究对象是更小年龄段的学生时,你会发现这个数字显著不同。
Wegge, Vandebosch 和 Eggermont(2014)\cite{wegge2014}
发现在1458名13--14岁的学生中,遭受网暴的学生数量要少得多。
这一结论与 Vanderbosch 和 Van Cleemput (2009)\cite{vander2009}
在2052名 12--18 的学生调查相似。在这项调查中, 11.1\% 的学生受到过网暴的
侵害。这一研究的结论是,网暴会随着学生年龄的增长而趋于普遍化。
Wegge 等人(2014)\cite{wegge2014}还注意到,30.8\% 也是传统暴力形式的受害者。

那么问题来了:为什么年少的学生受害会少。难道是因为他们比大学生拥有的
上网工具少,还是技术手段不够先进?进一步的问题是:加害者有哪些类型,
他们伤害他人的原因是什么。

\textbf{加害者类型}~~
在分析网暴时,试图理解具有进攻型倾向的这类人群是一个重要的因素。
首先应当讨论的是,在问及谁有攻击性时,是否是性别问题。
Slonje 和 Smith (2008) \cite{slonje2008}发现,男性比女性频度高得多。
他们还发现 36.2\% 的学生不知道加害者的性别。这很有意思,
因为对于每个人来说,施暴男性比例相同,更重要的是,超过1/3的学生
实际上不知道谁在欺负他们,这更增加了难以逃避的恐惧和阴影。

\textbf{受害者类型}~~
研究人员也对受害者进行了多项研究,这项研究又叫``网络受害学''。
\marginpar{cybervictomology}
Abeele 和 Cock (2013)\cite{abeele2013} 进行了一项研究,结论是,
受害人的性别依网暴形式而大相径庭。Abeele 等人发现,男生更容易成为
网暴的直接受害者,而女生则更容易成为网暴的间接受害者 (如流言传播)。
这些发现在如今社会似乎是这样,男性
表现出更多的直接对抗,而女性则是流言的载体。

对受害人性别的研究不多,倒是确有不少研究受害人的性格特征。
Faucher 等人 (2014)\cite{faucher2014} 发现,有很多原因使他们成为
网暴的对象,如长相、人际关系、 乃至不同的观点。
Davis, Randall, Ambrose 和 Orand (2015)\cite{davis2015}
也研究了受害者的统计数据。Davis 等人 (2015)\cite{davis2015}的一些结论
得出了成为受害者的另一些原因,他们发现 14\% 的受害者是因为他们的性取向。

这些结论非常重要,他们符合许多人印象中的传统欺凌模式,而如今被延伸到网络世界。
这将引出新的问题:二者有什么关系,网暴如何产生影响,侵扰如何持续。

欺凌和受害的关系也被重点研究。Beran 和 Li (2007)\cite{beran2007}
调查了432 个中学生,他们中将近一半受到过网暴和传统暴力的侵害。
多个研究证实了这个结论。Wegge 等人 (2014)\cite{wegge2014} 的结论也是,
那些受传统暴力侵害的人,也更容易成为网暴的受害者。研究结果显示的
另一个有趣的关系是,在网络世界,受害者也容易成为加害者。
Beran 等人 (2007)\cite{beran2007}确认了这个结论:``通过技术手段
受害的学生,容易使用技术手段加害他人''。
Faucher 等人 (2014)\cite{faucher2014} 也发现了相同的结果:
学生在网上欺负别人,是因为他们先被别人欺负。

加害者如何找到他的施暴对象,这项研究也有人做过。
\marginpar{我算术不好}
Wegge 等人 (2014)\cite{wegge2014} 研究了加害者的偏好,发现 27\%
与受害人在同一年级,14.2\%不在同一年级,大约49.6\% 不是同学。
这个证据多少跟其他研究有点冲突,
那些研究结论是,欺凌一般发生在学校或家中,近半数恶霸更喜欢欺负不跟
他们一起上学的人。这持续建立和加强了网暴思想,它允许恶霸塑造个人形象,
不经实质性的人身威胁就威胁影响到他人。

本综述第一部分专注于加害者与受害者的统计数据,现在我们将焦点放在网暴带给
受害人长期的影响和心理伤害,以及不同的网暴形式上。
尽管使用不同的平台,对受害人持续的影响非常相似。 Faucher 等人 (2014)
\cite{faucher2014}
的结论是,网暴的对大学学生一个主要影响是,一些作业不能完成。
而许多人认为,影响仅仅是压抑和自尊,与研究结果大不相同。
Beran 等人 (2007)\cite{beran2007}也从受害者发现类似的结果,网暴受害人称
他们不能在学校拿到足够的分数,注意力不能集中。这些发现表明网暴对
受害人的伤害比多数人在表面上看到的更甚。

Pieschl、Porsch、 Kahl 和 Klockenbusch (2013)\cite{pieschl2013}发现,
受害人在面临第二次网暴时痛苦减少。这说明受害人可能因时间流逝而对侵害
变得事实上的迟钝。

面对欺凌,受害人学会应对、从痛苦经历中走出,避免在人生和职业生活中长期
承受痛苦,是非常重要的。Davis 等人 (2015)\cite{davis2015}
研究了应对策略,他们把应对分成不同的两类:行为策略和认识策略。
Davis 等人 (2015)\cite{davis2015} 发现 74\% 的参与者更偏好行为策略。
在这74\% 中,69\% 发现这个策略是有效的。这些行为策略包括寻求社会支持、
情感倾泻或者无视、屏蔽欺凌。由于网暴呈增长趋势,人们开展了不同的项目,
帮助提高对网暴的认知,以及为受害人提供帮助。
其中一个为人所知的项目网暴2.0。
Garaigordobil 和 Martinez-Valderrey (2015)\cite{gara2015} 进行了研究,
发现这个项目对减少传统暴力和网暴都有效,而且更为重要的是,它增强了
同学对受害者的同情心。这是对抗欺凌的一大步,因为同龄人总是持续地互相影响的。
欺凌不是玩笑,是错误的,这一认识在同学之间成为共识,对受害者抱以同情,
会在改变社会行为上走得更长远。如果同学们对欺凌和受害人缺少关注,很可能
削弱这一行为。

代之以面对的另一个问题是,为什么不躲开那个处境,不给欺凌以机会呢?
\marginpar{这段没看懂}
Arntfield (2015)\cite{arntfield2015} 讨论了有关利用社交媒体的风险,
结论是,实质性回报
的得失,同参与者对权力、声望、性别、社会地位提升的期望没有直接联系。
在今日社会,对青春期和大学思想而言,这是一个准确的结论。
事实上,为了适应,被同龄人认为``酷'',你需要出现在社交媒体,理解许多
在同学之间谈论的话题。不管是
趋势标签、病毒式视频还是模因,
\marginpar{hashtag, viral video, memes这些我也不懂}
这些都在同龄人间分享讨论。学生可能在这些话题下遭受欺凌,
也可能因不知道在文化圈里的新知识遭受欺凌,这确实是个怪圈。

\section{结论}
网暴给了欺凌者大得多的选择空间来威胁受害者,这也可能是施害更为容易的原因。
所有不同的网暴方式中,
Faucher 等人 (2014)\cite{faucher2014}发现,最常见的平台是社交媒体,
短信和email,大约占半数时间,其次是blog 论坛和聊天室,约25\%。
社交媒体是网暴最常见的平台,这并不让人吃惊。它可以让欺凌行为对普通受害人来说
完全匿名。这让人可以创建假账号制造个人形象,欺凌他人。
很多研究也关注在使用社交媒体实施欺凌的关键因素,那就是,照片或者伤害性的
评论,会永远留在网络上。
Davis 等人 (2015)\cite{davis2015} 提到,他们收到一些回应,讨论关于
``传统欺凌方式如果出现在今天的数字时代,会被怎样放大''的问题。
Faucher 等人 (2014)\cite{faucher2014} 也谈及网暴比一般欺凌有更长的
``上架期''。这更为关键,因为网上侵犯性的材料会被重复不断地访问,
痛苦被持续地带给受害者,造成更多伤害。
社交媒体在网暴中非常普遍,研究也扩展到移动电话,它在网暴中扮演的角色。
Abeele 等人 (2013)\cite{abeele2013} 研究了利用移动电话欺凌的不同方面,
发现最普遍的方式是短信流言,其后是电话流言。他们还发现,女孩常传播流言,
而男孩更多通过移动电话发出威胁。这符合女性传播流言、男性更具体力进攻性这一
大众形象。这也很有意思,即使在网络世界,大众形象也与真实世界保持一致。


\bibliographystyle{unsrt}
\bibliography{bookref}

\end{document}
