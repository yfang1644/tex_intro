%%============================================================================
%%
%%       Filename:  table1.tex
%%
%%    Description:  thesis review
%%
%%        Version:  1.0
%%        Created:  09/08/2018 09:23:46 PM
%%       Revision:  none
%%
%%         Author:  Fang Yuan (yfang@nju.edu.cn)
%%   Organization:  nju
%%      Copyright:  Copyright (c) 2018, Fang Yuan
%%
%%          Notes:  
%%                
%%============================================================================
\documentclass[a4paper,12pt]{article}
\usepackage[fontset=none]{ctex}
\usepackage[paperwidth=15cm, paperheight=23.5cm,
            margin=0pt, top=3.5cm, bottom=1.5ex,
            left=2pt,right=2pt]{geometry}
\usepackage[a4,frame,center]{crop}
\usepackage{ltxtable}

\setCJKfamilyfont{caiyun}{STCaiyun}
\setCJKfamilyfont{songti}{HYZhongSongS}
\setCJKfamilyfont{kaiti}{AR PL UKai CN}
\setCJKfamilyfont{fangsong}{STFangsong}
\setCJKfamilyfont{heiti}{WenQuanYi Micro Hei}
\setCJKfamilyfont{xinwei}{STXinwei}
\setCJKfamilyfont{lishu}{LiSu}
\setCJKmainfont{SimSun}
%\setCJKmainfont[BoldFont={WenQuanYi Micro Hei}, ItalicFont={STFangsong}]{SimSun}
\setCJKsansfont{WenQuanYi Micro Hei}
\setCJKmonofont{AR PL UKai CN}

%\setmainfont{Times New Roman}
\setsansfont{Droid Sans}
\setmonofont{DejaVu Sans Mono}

\newcommand*{\songti}{\CJKfamily{songti}}   % 宋体
\newcommand*{\kaishu}{\CJKfamily{kaiti}}     % 楷书
\newcommand*{\fs}{\CJKfamily{fangsong}}   % 仿宋
\newcommand*{\heiti}{\CJKfamily{heiti}}     % 黑体
\newcommand*{\wei}{\CJKfamily{xinwei}}    % 新魏
\newcommand*{\lishu}{\CJKfamily{lishu}}   % 隶书
\newcommand*{\cy}{\CJKfamily{caiyun}}     % 彩云


\newcommand*\markC{%
\begin{picture}(0,0)
    \linethickness{1.3pt}\unitlength 1cm
    \put(0,0){\framebox(15,20)[c]{}}
\end{picture}}
        
\newcommand*\markinfo{
    \shortstack{
    {\renewcommand{\CJKglue}{\hskip 1em} \bf \Large 南京大学\vspace{5mm}}
        \\
    {\renewcommand{\CJKglue}{\hskip .4em}
       \bf \Large 研究生学位申请书 · 附表一\vspace{5mm}}
        \\
    \fontsize{16}{16}\selectfont
    \fs \large 指导教师对研究生学位(毕业)论文评语
    }
}
\cropdef[\markinfo]\relax\relax\markC\relax{istin}
\crop[istin]

\begin{document}
\pagestyle{empty}
\fontsize{12}{16}\selectfont
\begin{center}
\begin{tabularx}{\textwidth}{c|c|c|c|c|X|X}
研究生姓名 & \multicolumn{2}{c|}{佚名} & 入学时间 & 2013.03 & 攻读学位
        & 工程硕士\\\hline
所~学~专~业   & \multicolumn{2}{c|}{电子与通信工程}
        & 研究方向 & \multicolumn{3}{l}{空气管道}\\\hline
论~文~题~目 & \multicolumn{6}{l}{地月距离测量与算法}\\\hline
\end{tabularx}

\fontsize{12}{14}\selectfont
\vskip 8mm
\Large \kaishu 导师评语\end{center}

\fontsize{12}{12}\selectfont

在生活水平提高的同时,医疗、健康也越来越得到人们的关注。在这个背景下,
医疗一体机具有广阔的市场前景。论文作者在这一产品的生产过程中发现了
影响产品质量的一些关键性因素,并对此展开了研究,具有明确的研究背景和
实用价值。

论文详细分析了产品生产的现状,从``持续改善''的理念出发,探讨了改进生产
工艺和流程的一些措施,改进了测试软件,将原有的手工操作过程转向半自动化
过程,初步实现了降低成本、提高效率,使企业在激烈的市场竞争中保持优势。
其研究成果得到了企业的肯定。
    
论文选题符合专业培养目标,能够达到综合训练目标,时效性较强。作者在
选题和论文撰写过程中,查阅了大量相关文献,结合工作经验,灵活运用已学
知识解决课题中遇到的问题。论文表现出作者具有扎实的理论基础,具备独立
从事科学研究的能力,工作细致踏实,具有团结协作精神。

本人认为论文达到硕士学位论文水平,同意该生进行硕士研究生论文答辩和
硕士学位申请。

\vfill
\hfill {\heiti 指导教师}\underline{\hskip 35mm} (签章)\qquad\qquad
\vskip 1cm
\hfill 年\hskip 12mm 月\hskip 12mm 日\qquad\qquad

(可加附页)

\end{document}
