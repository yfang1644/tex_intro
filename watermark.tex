%%============================================================================
%%
%%       Filename:  watermark.tex
%%
%%    Description:  
%%
%%        Version:  1.0
%%        Created:  09/28/2018 08:09:36 AM
%%       Revision:  none
%%
%%         Author:  Fang Yuan (yfang@nju.edu.cn)
%%   Organization:  nju
%%      Copyright:  Copyright (c) 2018, Fang Yuan
%%
%%          Notes:  
%%                
%%============================================================================
\documentclass[a4paper]{article}
%不同环境,中文字体配置可能会有不同
\usepackage[fontset=ubuntu]{ctex}
%右边多留出1cm,作为边注
\usepackage[left=3cm,right=4cm]{geometry}
\usepackage{amssymb}
\usepackage{xcolor}   % 彩色文本

\title{音频数字水印技术的发展}

\author{$\blacksquare \square \blacksquare$\\[2ex]
 南京大学 电子科学与工程学院}

\begin{document}
\maketitle

\begin{abstract}
数字水印(Digital Watermark)是在数据中嵌入隐蔽标记的一种技术。
它在音像信息防伪、加密、知识产权保护、以及法律证据确认等方面都有重要的应用。
\footnote{本文摘自2018年我院某本科论文的第一章。作为文献综述,结构上
    缺少结论部分。结论应概括研究的工作重点及目标,它应以对已有信息
    正反两方面的分析为基础。编者对原文略有修改,题目为编者所加。}

\end{abstract}

关键词:~~数字水印~~掩蔽效应~~鲁棒性
\section{引言}
随着通信技术和互联网领域的发展,网络资源的带宽不断加大,获取数字媒体的
渠道不断扩大,人与人之间信息的分享不断得到促进。现如今,互联网产业已经
成为整个社会的投资热点,也是社会在信息化进程中的重要推动力。信息以数据的
形式在互联网上流动,主要包含文字、音频、图像和视频等。

\marginpar{\color{red}{分析面过窄}}
与传统的信息传播方式相比,这些在网络上的数字数据具有高保真、易存储、
易传输、易复制等的特点。这些成果可以轻易地在极短时间内被修改、复制、传播到
世界的各个地区,在有效地促进了信息传播的同时,也极为严重地威胁到艺术创作者
和版权所有者的利益,对版权人的劳动价值带来了极大的侵害。

数字水印技术利用人类感觉器官的不敏感特性及数字信号本身的冗余,将带有一定
特征的信号嵌入到某个载体中,并在一定条件下从宿主信号中提取,从而实现对
该载体的确认。这一技术在信息安全、知识产权保护等方面有着重要的应用。

\section{研究发展}
\marginpar{\color{red}{Chronological}}
数字水印一词最早由 Van Schyndel 等人\cite{schyndel1994}于1994年提出,
指在图像上编码携带有认证或授权码信息的“不可检测”水印的方法。随着该技术
近30年的完善和研究方法的不断优化,数字水印的应用领域从最初的图像逐步
扩展到音频视频等范围,应用目的也不局限于认证等,而是逐渐拓展到
加密通讯、来源分析等。数字水印如今指的是把我们需要隐藏的内容信息与某种
载体如图像、音频和视频等相结合,使载体不受水印影响的前提下可由接收对象
通过某种方法提取出水印进行鉴别,并能保证在传播的过程中不易被修改破坏。
在互联网信息时代,数字水印技术已经成为应对版权问题的一种极为可行的方法。

伴随着社会信息化进程的加快和数字媒体的快速进步,数字水印技术的优化也
得到了极大的推动。自从数字水印的概念提出后,该方法已经从图像逐步拓大到
音频、视频等范畴。1994年Van Schyndel 等人\cite{schyndel1994}定义
数字水印概念的同时,也发明了一种在图像空间域中基于
LSB (Least Significant Bit,最低有效位)的水印嵌入算法,具备良好的安全性。
但是这种水印算法的稳定性不高,容易被修改破坏。他们还提出了一种
在最低有效位上叠加水印的算法,提高了安全性,但使得解码的难度变得更高。
1997年,Cox 等人\cite{cox1997} 提出了用于图像但可以推广到音频、视频和
多媒体数据领域的数字水印方法,该方法基于扩频的思想,通过把水印变换成
独立同分布的随机高斯向量再插入到原始图像中,高斯噪声的使用使得水印对数字
信号处理攻击如有损压缩、滤波、数模模数转换(ADC/DAC)等具有较好的鲁棒性,
该方法已经成为数字水印算法的一种典范方案。Bender 等人\cite{bender1996}
实现了分别基于最低有效位、相位编码、扩展频带和回声隐藏的数字水印方案,
通过实验证明了这些算法的利用价值,并对其性能做了一定的分析。
同年,Boney 等人\cite{boney1996}将 Cox 的算法应用到数字音频的领域,
选取一个伪随机序列执行多级滤波,而后按照局部音频的强度给滤波序列加以
不同的权重,使得音频强度低的位置嵌入能量较小的数字水印,很好地利用了
人耳听觉系统的掩蔽特点,将音频有损压缩为MP3格式的文件对该方法影响不大。

由于数字信号在变换域上比时间域上拥有更好的抗攻击性能,变换域方案
成为专家学者的研究重点。1996年,Bender 等学者\cite{bender1996} 提出
一种基于相位编码的算法,把音频分段后将音频片段进行离散傅里叶变换处理,
计算出幅度和相位后按原来信号的相位差和相位序列构建新的相位序列,
之后做傅里叶逆变换处理后即可得到已经加入数字水印的音频。
Yang Yan 等人\cite{yang2009}实现一种基于DCT(Discrete Cosine Transform,
离散余弦变换)的方案,他们把音频分块后对每个音频块进行 DCT 变换,
通过调制DCT交流系数的方法完成了水印的插入,对低通滤波、加入噪声的攻击
形式具体现出良好的性能。为了进一步增加水印的鲁棒性和安全性,
Elshazly 等人\cite{elshazly2012}实现了一种在DWT (Discrete Wavelet
Transform,离散小波变换)域方案,他们对音频进行DWT后,在DWT低频分量中
插入信息,取得了令人满意的效果。

国内近年来在该领域获得了十分显著的成果。2001年,王秋生等人\cite{wang2001}
实现了在DCT域中通过修改中频系数插入数字水印的方案,并且把插入的数字水印
由没有实际意义的PN码(Pseudo-Noise Code,伪噪声序列)改成了二维的图像信息。
2004年,王向阳、杨红颍等人\cite{wang2004}先将水印图像随机置乱后再自适应地
嵌入离散余弦变换域的高频DCT系数当中,提高了水印的安全性和不可见性。
高海英和钮心忻等人\cite{gao2005}提出了一种在小波变换域内基于量化的自同步算法,
通过对小波域的低频分量执行量化的方法插入水印,并设置了可供算法进行同步的
标志来确定数字音频水印的插入位置。

\bibliographystyle{unsrt}
\bibliography{bookref}

\end{document}
