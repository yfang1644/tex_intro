%%
%%       Filename:  study.tex
%%
%%    Description:  
%%
%%        Version:  1.0
%%        Created:  10/30/2019 03:02:54 PM
%%       Revision:  none
%%
%%         Author:  Fang Yuan (yfang@nju.edu.cn)
%%   Organization:  nju
%%      Copyright:  Copyright (c) 2019, Fang Yuan
%%
%%          Notes:  
%%                
%%============================================================================
\documentclass[10pt,t]{beamer}
\usepackage[fontset=none]{ctex}
\usepackage{graphicx}
\input{style.tex}

\setCJKfamilyfont{caiyun}{STCaiyun}
\setCJKfamilyfont{songti}{HYZhongSongS}
\setCJKfamilyfont{fangsong}{STFangsong}
\setCJKfamilyfont{heiti}{WenQuanYi Zen Hei}
\setCJKfamilyfont{kaiti}{AR PL UKai CN}
\setCJKfamilyfont{xinwei}{STXinwei}
\setCJKfamilyfont{lishu}{LiSu}
\setCJKmainfont[BoldFont={WenQuanYi Zen Hei}, ItalicFont={STFangsong}]{WenQuanYi Zen Hei}
\setCJKsansfont{WenQuanYi Zen Hei}
\setCJKmonofont{AR PL UKai CN}

\newcommand*{\song}{\CJKfamily{songti}}   % 宋体
\newcommand*{\fs}{\CJKfamily{fangsong}}   % 仿宋
\newcommand*{\hei}{\CJKfamily{heiti}}     % 黑体
\newcommand*{\kai}{\CJKfamily{kaiti}}     % 楷书
\newcommand*{\wei}{\CJKfamily{xinwei}}    % 新魏
\newcommand*{\lishu}{\CJKfamily{lishu}}   % 隶书
\newcommand*{\cy}{\CJKfamily{caiyun}}     % 彩云

\lecture[1]{Electronic Science \& Engineering}{ESE}

\subtitle{\hspace{1.5em}Departments and Majors}

\date{}

\begin{document}
\begin{frame}
  \maketitle
\end{frame}

\begin{frame}\frametitle<presentation>{Contents}
  \tableofcontents
\end{frame}

\section{学科分布}
\begin{frame}{Science and Engineering}
\centering
    \fontsize{12}{16}\selectfont
\begin{tabular}{ll}\hline
    一级学科  &  二级学科\\\hline
    物理      & 无线电物理\\
    电子科学与技术 & 物理电子学 \\
                   & 电路\\
                   & 微电子\\
                   & 电磁场、微波\\
    信息/通信工程  & 信号与信息处理\\
                   & 通信系统\\
    生物医学工程   &\\\hline
\end{tabular}
\end{frame}

\section{电子工程系}
\begin{frame}{Electronic Engineering}

    研究物理过程和电磁现象的基本规律, 开发新型的电子器件和系统
\begin{itemize}
    \item 无线电物理学
    \item 电磁场与微波技术 (THz)
    \item 超导
\end{itemize}
    
\end{frame}

\section{微电子与光电子学系}
\begin{frame}{Microelectronics and Optical Electronics}
\begin{itemize}
    \item 半导体材料与器件
    \item 微结构光电子材料与器件
    \item 集成电路设计
\end{itemize}
\end{frame}

\section{通信工程系}
\begin{frame}{Communucation Engineering}
\begin{itemize}
    \item 宽带网络通信\\
        异构网络融合、系统跨平台处理、网络优化与管理、通信协议转换、
        软交换与高级信令、移动自组织网络、自适应编码调制,多天线,OFDM,
        智能交通系统、车联网

    \item 多媒体安全通信 --- 计算机网络安全

    \item 雷达信号处理 (信号、信息处理)\\
        相控阵信号处理、目标识别

    \item 智能信息处理 (信号、信息处理)\\ 
        信号处理、人工智能、模式识别、网络通信
\end{itemize}
\end{frame}

\section{信息电子学系}
\begin{frame}{Information Electronics}
\begin{itemize}
    \item 嵌入式软件研发、图像工程\\
        图像控制与作业自动化、视频显示及处理、
        嵌入式操作系统及嵌入式手持设备应用软件开发、
        无线自组织网络及无线传感器网络算法与设备

    \item 先进器件与信息功能材料实验室\\
        半导体材料、器件物理与工艺

    \item 微纳集成技术\\
        超大规模集成电路与芯片应用、半导体器件、光电传感器、
        微电子器件在移动存储、光电信号探测/存储/转换/传输的集成应用技术

    \item 影像处理技术\\ 
        3-D 显示、编解码、
        立体影像处理、数字远程医疗、立体化作战指挥控制等

    \item 电磁材料与应用\\
        纳米结构、金属超细粉体、铁氧体材料、磁性薄膜的制备、微波测量技术,
        研究成果可应用于军事装备隐身和民用防电磁辐射材料
        
    \item 生物医学电子\\
        生物医学电子学、生物医学超声 (B超、非线性声学)、
        微波医学和生物医学信息检测与处理
\end{itemize}
\end{frame}
\end{document}
