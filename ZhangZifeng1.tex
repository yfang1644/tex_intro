%%============================================================================
%%
%%       Filename:  table1.tex
%%
%%    Description:  thesis review
%%
%%        Version:  1.0
%%        Created:  09/08/2018 09:23:46 PM
%%       Revision:  none
%%
%%         Author:  Fang Yuan (yfang@nju.edu.cn)
%%   Organization:  nju
%%      Copyright:  Copyright (c) 2018, Fang Yuan
%%
%%          Notes:  
%%                
%%============================================================================
\documentclass[a4paper,12pt]{article}
\usepackage[fontset=none]{ctex}
\usepackage[paperwidth=15cm, paperheight=23.5cm,
            margin=0pt, top=3.5cm, bottom=1.5ex,
            left=0pt,right=0pt]{geometry}
\usepackage[a4,frame,center]{crop}
\usepackage{ltxtable}

\setCJKfamilyfont{caiyun}{STCaiyun}
\setCJKfamilyfont{songti}{HYZhongSongS}
\setCJKfamilyfont{kaiti}{AR PL UKai CN}
\setCJKfamilyfont{fangsong}{STFangsong}
\setCJKfamilyfont{heiti}{WenQuanYi Micro Hei}
\setCJKfamilyfont{xinwei}{STXinwei}
\setCJKfamilyfont{lishu}{LiSu}
\setCJKmainfont{SimSun}
%\setCJKmainfont[BoldFont={WenQuanYi Micro Hei}, ItalicFont={STFangsong}]{SimSun}
\setCJKsansfont{WenQuanYi Micro Hei}
\setCJKmonofont{AR PL UKai CN}

%\setmainfont{Times New Roman}
\setsansfont{Droid Sans}
\setmonofont{DejaVu Sans Mono}

\newcommand*{\songti}{\CJKfamily{songti}}   % 宋体
\newcommand*{\kaishu}{\CJKfamily{kaiti}}     % 楷书
\newcommand*{\fs}{\CJKfamily{fangsong}}   % 仿宋
\newcommand*{\heiti}{\CJKfamily{heiti}}     % 黑体
\newcommand*{\wei}{\CJKfamily{xinwei}}    % 新魏
\newcommand*{\lishu}{\CJKfamily{lishu}}   % 隶书
\newcommand*{\cy}{\CJKfamily{caiyun}}     % 彩云

\newcommand*\markC{%
\begin{picture}(0,0)
    \linethickness{1.3pt}\unitlength 1cm
    \put(0,0){\framebox(15,20)[c]{}}
\end{picture}}
        
\newcommand*\markinfo{
    \shortstack{
    {\bf \Large 南~京~大~学\vspace{5mm}}
        \\
    {\renewcommand{\CJKglue}{\hskip .4em}
       \bf \Large 研究生学位申请书$\cdot$附表一\vspace{5mm}}
        \\
    \fontsize{16}{16}\selectfont
    \fs \large 指导教师对研究生学位(毕业)论文评语
    }
}
\cropdef[\markinfo]\relax\relax\markC\relax{istin}
\crop[istin]
\addtolength{\leftskip}{1em}
\addtolength{\rightskip}{1em}
    
\begin{document}
\pagestyle{empty}
\fontsize{12}{16}\selectfont
\begin{center}
\begin{tabularx}{\textwidth}{c|c|c|c|c|X|X}
研究生姓名 & \multicolumn{2}{c|}{张梓峰} & 入学时间 & 2016 & 攻读学位
        & 工程硕士\\\hline
所~学~专~业   & \multicolumn{2}{c|}{电子与通信工程}
        & 研究方向 & \multicolumn{3}{l|}{音频信号处理}\\\hline
论~文~题~目 & \multicolumn{6}{l|}{基于音频信号的高带宽数据传输系统的设计与实现}\\\hline
\end{tabularx}
\vskip 8mm
\Large \kaishu 导师评语\end{center}

\fontsize{12}{14}\selectfont

在传统的汽车生产中, 车载软件的更新是一项比较繁琐的工作。利用车载
模拟音频系统进行数字信号的传输, 具有一定的实用价值。其需要解决的技术
要点主要集中在提高数字传输速度和可靠性上。

论文根据音频系统的需求, 研究了音频信号数据传输的关键技术, 包括数字
频率合成和频谱分析技术等, 进而基于声音的时频特性, 提出一种新的编解码
方案,利用不同的频率和幅度表示不同的数据。经过不断改进, 实现了数字传输
的功能。

作者广泛阅读了相关文献,对研究和应用现状进行了分析和综述,开展了有
价值的研究工作。论文的主要有特色的工作和取得的成果如下:

1. 研究了双音多频率, FSK 和曼彻斯特通信编码技术;

2. 设计了一套用于软件更新的通讯协议;

3. 使用签名验证的防止篡改技术, 用哈希校验和循环冗余校验确保数据传输的正确。

设计方案在PC上进行了模拟验证, 实现了主要功能, 单声道数据传输性能可以
达到14.3kbps。

论文反映作者已掌握了相关专业的基础理论知识,具有独立从事科研开发的能力。
论文叙述清楚、研究方法科学,结果可信, 达到了工学硕士学位论文的水平。
同意安排硕士学位论文答辩和学位申请。

\vfill
\hfill {\heiti 指导教师} \underline{\hskip 35mm} (签章)\qquad\qquad
\vskip 3mm
\hfill 年\hskip 12mm 月\hskip 12mm 日\qquad\qquad

\noindent(可加附页)

\end{document}
