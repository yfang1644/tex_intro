% $Header: /Users/joseph/Documents/LaTeX/beamer/solutions/conference-talks/conference-ornate-20min.en.tex,v 90e850259b8b 2007/01/28 20:48:30 tantau $

\documentclass[14pt,t]{beamer}
\usepackage[fontset=ubuntu]{ctex}
\usepackage[english]{babel}
\geometry{papersize={160mm,90mm}}
%\usepackage{times}
%\usepackage[T1]{fontenc}


\mode<presentation>
{%
  \usetheme{Warsaw}%
  \setbeamercovered{transparent}%
}

\title[电子实践导学] % (optional, use only with long paper titles)
{电子实践导学}

\subtitle{如何做实验/撰写实验报告}

% (optional, use only with lots of authors)
\author[Fang Yuan (yfang@nju.edu.cn)]
{方元 (yfang@nju.edu.cn)}

\institute[南京大学电子学院] % (optional, but mostly needed)
{南京大学电子学院}

\date[expcourse] % (optional, should be abbreviation of conference name)
{\today}

\subject{Theoretical Computer Science}
% This is only inserted into the PDF information catalog. Can be left
% out. 

% If you have a file called "university-logo-filename.xxx", where xxx
% is a graphic format that can be processed by latex or pdflatex,
% resp., then you can add a logo as follows:

\pgfdeclareimage[height=7mm]{university-logo}{njulogo.pdf}
\logo{\pgfuseimage{university-logo}}

% Delete this, if you do not want the table of contents to pop up at
% the beginning of each subsection:
%\AtBeginSubsection[]%
%{%
%  \begin{frame}<beamer>{Agenda}
%    \tableofcontents[currentsection,currentsubsection]
%  \end{frame}
%}

% If you wish to uncover everything in a step-wise fashion, uncomment
% the following command: 

%\beamerdefaultoverlayspecification{<+->}

\begin{document}

\begin{frame}
  \titlepage
\end{frame}

\begin{frame}{Agenda}
  \tableofcontents
  % You might wish to add the option [pausesections]
\end{frame}


% Structuring a talk is a difficult task and the following structure
% may not be suitable. Here are some rules that apply for this
% solution: 

% - Exactly two or three sections (other than the summary).
% - At *most* three subsections per section.
% - Talk about 30s to 2min per frame. So there should be between about
%   15 and 30 frames, all told.

% - A conference audience is likely to know very little of what you
%   are going to talk about. So *simplify*!
% - In a 20min talk, getting the main ideas across is hard
%   enough. Leave out details, even if it means being less precise than
%   you think necessary.
% - If you omit details that are vital to the proof/implementation,
%   just say so once. Everybody will be happy with that.

\section{Why Do We Need Experiments}

\subsection{Science is a way we understand the world}

\begin{frame}{Scientific Methods}{}
    How we understand our world?
\begin{itemize}
    \item Depending on our experience and intuition.
    \item Depending on reasoning and empirical knowledge.
\end{itemize}

    Science is a way of learning about \alert{what} is in the natural world,
    \alert{why} it works, and \alert{how} it got to be the way it is.
    \ \\ \ \\

    It is not simply a collection of facts, rather it is a path to
    understanding. 

\end{frame}

\begin{frame}{Limitations of Science}
Science has limits:
\begin{itemize}
    \item Science doesn't make moral judgments.
    \item Science doesn't make aesthetic judgments.
    \item Science doesn't tell you how to use scientific knowledge.
    \item Science doesn't draw conclusions about supernatural explanations.
\end{itemize}
    Besides, science may go wrong. But still science is a world-wide
    accepted method that we rely on.
\end{frame}

\iffalse
\begin{frame}{Scientific Literacy Training}
\centering
\includegraphics[width=.9\textwidth]{fig2}

Sampling Survey on the Scientific Literacy in China
\end{frame}

\begin{frame}{QUIZ-1}
    \small
\begin{itemize}
    \item 我们呼吸的氧气来源于植物. (T/F)
    \item 激光是通过汇聚声波产生的. (T/F)
    \item 所有的放射性现象都是人为的. (T/F)
    \item 人类是由早期的动物演化而来的. (T/F)
    \item 第一手资料不应借助于
        
        A. 实地调查. B.查找原始资料. C.听取专家评论.
\end{itemize}
\end{frame}
\fi

\subsection{Experiments' Position in Research}

\begin{frame}{Procedure of a Scientific Research}
\begin{center}
    \includegraphics[width=.75\textwidth]{process.pdf}
\end{center}

    experiment(实验) vs. test(试验).  ex-per-iment. 

    experiments $\to$ experience $\to$ expert.
\end{frame}

\begin{frame}{Types of Experiment}
\begin{itemize}
    \item Confirmatory Experiments (验证型实验)

        To test some relatively simple hypothesis stated a priori.
        This includes designing experiment(设计实验) or so-called
        comprehensive/synthesized/combined experiment (综合实验).

        Most college experiments (esp. in first 2 years) are confirmatory.
    \item<2-> Exploratory Experiments (探索型/研究型实验)

        To generate data with which to develop hypotheses for future
        testing. They may ``\alert{work}'' or ``\alert{not work}''.
\end{itemize}
\end{frame}

\subsection{Why Do We Need Experiments}
\begin{frame}{Why Do We Need Experiments}
        \only<2>{
\begin{itemize}
    \item For credit.
    \item To understand the theory better.
    \item To find errors in the theory.
    \item To learn more knowledges.
    \item To get operation skills.
    \item To get innovating ideas.
    \item To improve scientific literacy.
\end{itemize}
    }
\end{frame}

\iffalse
\begin{frame}{QUIZ-2}
    When choose Other, please specify.
\small
\begin{itemize}
    \item 你需要研究在海拔4000米的青藏高原上水的沸点, 这个实验属于:
        
        A.验证型实验.  B.研究型实验.
    \item 当某项条件影响结果时, 应尽量发挥它的作用, 以便结果更符合预期.
        (T/F/O)
    \item 在一个研究型实验中, 如果实验结果与预期不符, 则实验结果不宜公开.
        (T/F/O)
    \item 你如何理解``实验结果''和``结论''之间的关系?

        A. 客观/主观. B. 定量/定性. C. Both. D. Neither.
\end{itemize}
\end{frame}
\fi

\section{How to Perform an Experiment}

\subsection{Before Experiment}
\begin{frame}{Preparation}
    An experiment can't be random/aimless.
\begin{itemize}
    \item You need to know the aim of the experiment.
        (What do you expect from the experiment.)

        Confirmatory experiments are based on complete theory.
    \item You need to know the objects.
        (What materials/devices you use, the performance/specifications/... )
    \item You have rough expectations to the results.
\end{itemize}

\end{frame}

\subsection{The Procedure}
\begin{frame}{During Experiment}
\begin{itemize}
    \item Operations.
    \item Data gathering (reliable, objective, always keep raw data).
    \item Keep curiosity, questioning.
    \item Device failures belong to experiment.
\end{itemize}
    You need to know exactly what you did and for what purpose you did.
\end{frame}

\begin{frame}{Observations}
Observation beyond our eyes.

How much we rely on our observation.
    \begin{center}
\begin{minipage}[c][4cm][c]{.4\textwidth}
    \includegraphics[width=.8\textwidth]{meter.jpg}

    \includegraphics[width=.8\textwidth]{generator.jpg}
\end{minipage}
\begin{minipage}[c][4cm][c]{.4\textwidth}
    \includegraphics[width=.9\textwidth]{oscilator.png}
\end{minipage}
    \end{center}

``Do we see through a microscope?'' (Ian Hacking, 1981)
\end{frame}

\section{Lab Report(Scientific Report)}

\subsection{Layout/Structure}
\begin{frame}{Scientific Report}
\begin{itemize}
    \item To communicate with others.
    \item The experiment must be repeatable.
    \item To show your understanding and your thinking(lab report).
\end{itemize}

The format of a report should be:
\begin{itemize}
    \item neat
    \item formatted
    \item easy to understand
\end{itemize}
\end{frame}

\begin{frame}{Layout of a Scientific Report}
\centering
\includegraphics[width=.9\textwidth]{fig1.png}
\end{frame}

\begin{frame}{Format of a Lab Report}
\begin{itemize}
    \item Title page (optional).
    \item Table of contents (optional).
    \item Divide report into logical sections.
    \item Draw conclusions from results.
\end{itemize}
    You finished a experiment in 2 hours. You have another 2 hours
    to write a paper, not a book.
\end{frame}

\begin{frame}{Important Factors}
To write a high quality research paper:
\begin{itemize}
    \item Coherence
    \begin{itemize}
        \item Include all necessary information in each section.
        \item Do not repeat information unless necessary.
    \end{itemize}
    \item Orgainization
    \begin{itemize}
        \item IMRD (journal-specific) structure.
        \item Put the right parts in the right place.
    \end{itemize}
    \item Relevance
    \begin{itemize}
        \item Confirm to length guidelines, keep paper focused.
        \item Choose a limited amount of data to present in each section.
    \end{itemize}
\end{itemize}

\end{frame}

\subsection{Sections in a Scientific Report}
%\begin{frame}[allowframebreaks]
%\frametitle<presentation>{Sections in a Scientific Report}
\def\sout#1{\vbox{\vskip 1ex
    {\color{red} \hrule height 0pt depth 1pt} \vskip -1ex \hbox{#1}}}
\begin{frame}{Sections in a Scientific Report}


\begin{itemize}
    \only<1>{\item Title
\begin{itemize}
    \item Highlight purpose, methods, scope
    \item Highly summarized
\end{itemize}
        Usually, the title of a lab report is often given.
        }

    \only<2>{\item Abstract

        The abstract paragraph includes \alert{purpose}, \alert{methods},
        \alert{key findings}, major \alert{conclusions}.

        You don't need to \sout{discuss} and \sout{interpret} something.

        The audience wants to know if this report worth of further reading.
        \ \\ \ \\

        Several keywords (3--5) are required in scientific report as
        search indexes. The keywords need to be specific.
        }

    \only<3>{\item Introduction
\begin{itemize}
    \item State the purpose of the experiment, the background and theory.
    \item Literature review required.(better recently, direct references)
    \item Instead of \sout{coping}, you may \alert{cite} other conclusions
        and results, and \alert{refer} to literatures.
\end{itemize}
        Introduction section may include the importance of the experiment
        and description of specialized equipment.
        }

    \only<4>{\item Methods/Procedure
    \begin{itemize}
        \item How materials were prepared.
        \item How measurements were made.
        \item Under what conditions the experiment was performed.
            What hypotheses you made.
        \item Analysis methods.
    \end{itemize}
        The procedure should be in chronological order.
        Write in detail as possible as you can, so that others can
        \alert{replicate your study} exactly by what you write in
        this section.
        (Assume the reader has no knowledge of what you did.)
        }
    \only<5>{\item Results

        The results need to be clear and logical.

\begin{itemize}
    \item Use \alert{tables}, \alert{charts}, \alert{figures}
        and \alert{diagrams}.
    \item Explain the results. (What do the results mean?
        can we trust them?...)
    \item Tables and figures need to be in standard format.
        (coordinates, units, ...)
\end{itemize}
        Keep your original records.\\
        \alert{NO NOT DROP ANY NEGATIVE RESULTS INTENTIONALLY.}
        }
    \only<6>{\item Discussion/Analysis

    State your aim/question/research.
    \begin{itemize}
        \item Whether the results is as expected?
        \item Analyze experimental error. (How accurate are the results,
            which factors impact the results.)
        \item Discuss both positive and negative findings.
        \item Limitations, ambiguities.
        \item Further research (how to improve/revise results,...).
        \item Critics, complains.
    \end{itemize}

        Present your \alert{opinion} based on \alert{facts}
        (experiment results). May also formulate new hypotheses.

        }
    \only<7>{\item Conclusion

\begin{itemize}
    \item Restate the question you were studying. (to remind the audience)
    \item summarize the findings.
        (answer the question "Was the problem solved?")
    \item Suggest further research.
\end{itemize}
    Conclusion is qualitative, while results are more quantitative.
    }

    \only<8>{\item References/Appendices
        
        List only references that were referred to.

        journals, monographs, theses, textbooks, media,...

        Raw data, calculations, graphs pictures or tables that have not
        been included in the report should be listed here.
    }
\end{itemize}
\end{frame}

\iffalse
\begin{frame}{QUIZ-3}
\small
\begin{itemize}
    \item 如果研究报告采用``IMRD''结构, 则下面哪些内容不宜写在``结果''中:
        A. 不符合预期的结果. B. 他人的结果.
        C. 对改善结果的建议.
    \item 在``实验方法(Methods)''节, 允许隐藏导致实验结果的关键信息,
        以保护商业机密. (T/F/O)
    \item 对于不符合预期的测量结果 (确认不是测量错误), 较好的做法是:
        
        A. 剔除它, 以便提高结果的质量. B. 当作普通数据对待.
        C. 基于此结果对理论提出修正. D. 其他(请给出你的意见).
    \item 请列出两个你做过的、印象深刻的实验, 它们基于的原理、
        实验目的和结果 (请使用表格形式).
\end{itemize}
\end{frame}
\fi

\begin{frame}
\frametitle<presentation>{Summary}
  % Keep the summary *very short*.
  \begin{itemize}
  \item Why do we need experiments.
  \item How to perform a experiment.
  \item How to write a lab report.
  \end{itemize}
\end{frame}

\appendix

% All of the following is optional and typically not needed. 
\section<presentation>*{\appendixname}
\subsection<presentation>*{For Further Reading}

\begin{frame}[allowframebreaks]
\frametitle<presentation>{For Further Reading}

\begin{thebibliography}{10}

\beamertemplateonlinebibitems
% Followed by interesting articles. Keep the list short. 

\iffalse
\bibitem{undsci}
    Max G. Bronstein, Roy Caldwell, et al.
    \newblock {\em Understanding Science}
    \newblock \url{https://undsci.berkeley.edu/index.php}

\bibitem{Franklin}
    Allan Franklin <allan.franklin@colorado.edu>
    Slobodan Perovic <sperovic@f.bg.ac.rs>,
    \newblock {\em Experiment in Physics},
    \newblock \url{https://plato.stanford.edu/entries/physics-experiment/}
\beamertemplatearticlebibitems
\bibitem{HeWei}
    何薇, 张超, 高宏斌,
    \newblock {\em 中国公民的科学素质及对科学技术的态度},
    \newblock 科普研究, 2008年12月, 第3卷

\beamertemplatearticlebibitems
\bibitem{Loyalka}
    Prashant Loyalka, Ou Lydia Liu, et al.
    \newblock {\em Skill levels and gains in university STEM education
        in China, India, Russia and the United States},
    \newblock \url{https://www.nature.com/articles/s41562-021-01062-3}
    (in abstract). \\
    中文报道: 美中俄印共同研究发现:中国大学生批判性思维能力下降, \url{https://www.163.com/dy/article/G43PSRPV0514DTKM.html}

\beamertemplatebookbibitems
\bibitem{Catherine}
    Catherine E. Snow , Kenne A. Dibner
    \newblock {\em SCIENCE LITERACY -- Concepts, Contexts, and Consequences},
    \newblock \url{https://www.ncbi.nlm.nih.gov/books/NBK396088/},
    A report of The National Academies of Science.Engineering.Medicine, 2016
\fi

\beamertemplateonlinebibitems
\bibitem{Martin}
    Greg Martin, \newblock{\em How to Write a Scientific Report},
    \newblock \url{https://youtube.com/watch?v=Vky9PDKx5KU}
\beamertemplateonlinebibitems
\bibitem{Apologia}
    Sherri Seligson,
    \newblock{\em How To Write A Lab Report},
    \newblock \url{https://homeschool-101.com/wp-content/uploads/2017/04/How-to-Write-a-Lab-Report-1.pdf},\\
    also find the video in \url{https://www.youtube.com/watch?v=FR28zf2Aiwo}

\bibitem{labwrite}
    Eric N. Wiebe, Catherine E. Brawner, et al.
    \newblock{\em The LabWrite Project: Experiences reforming lab
        report writing practice in undergraduate lab courses},
    \newblock Proceedings of the 2005 American Society for Engineering
        Education Annual Conference \& Exposition,
        \url{https://labwrite.ncsu.edu/}

\end{thebibliography}
\end{frame}

\iffalse
\begin{frame}{Pendulum}
    单摆实验中, 测量周期的时间点选取, 理论基础是下面的公式

    $$  \theta (t) = A\sin (\omega t) $$

   $$ \Delta \theta = A\omega \cos(\omega t) \Delta t$$ 
\end{frame}

\begin{frame}{Pendulum}
利用单摆测量重力加速度, 误差的定量分析

  $$  g = 4\pi^2 \frac{l}{T^2} $$

\begin{eqnarray}
    \Delta g&=&4\pi^2 \frac{\Delta l}{T^2}-8\pi^2 \frac{l \Delta T}{T^3}\\
     &=& g \frac{\Delta l}{l} - 2g \frac{\Delta T}{T} \\
     &\le& g \left|\frac{\Delta l}{l}\right|
        + 2g \left|\frac{\Delta T}{T} \right|
\end{eqnarray}
\end{frame}
\fi

\end{document}
