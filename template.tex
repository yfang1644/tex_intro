%%%%%%%%%%%%%%%%%%%%%%%%%%%%%%%%%%%%%%%%%%%%%%%%%%%%%%%%%%%%%%%%
%    本文档尽可能提供最小化tex模板,帮助初学者尽快适应
%    '%' 是单行注释符
%%%%%%%%%%%%%%%%%%%%%%%%%%%%%%%%%%%%%%%%%%%%%%%%%%%%%%%%%%%%%%%%
\documentclass{article}
% 引用 'article' 模板,缺省格式是:A4纵向幅面,单面(即不区分奇偶页)
\usepackage[fontset=none]{ctex}
% 中文字体包,选项 fontset=none 表示无系统定义字体,须自己配置,
% 配置方法见下面的 setCJKmainfont 命令。
% windows系统可以用 fontset=winfont,ubuntu系统可以用fontset=ubuntu,
% 省去自己配置字体工作

\setCJKmainfont[BoldFont={文泉驿等宽正黑},
                ItalicFont={AR PL UKai CN}]{AR PL SungtiL GB}
% 中文字体设置:主文档使用宋体,斜体环境用楷体,粗体用黑体

%\setmainfont{Times New Roman}
%\setsansfont{Droid Sans}
%\setmonofont{DejaVu Sans Mono}
% 英文字体通常不需要自己设置。如果想自己设置,就像上面三行那样

\usepackage[left=1in,right=1in,top=1in,bottom=1in]{geometry}
% 页面布局设置,上下左右各留1英寸边距
\usepackage{graphicx}     % 如果有插图,需引入graphicx包
\usepackage{listings}     % listings包用于排版程序清单
%%%%%%%%%%%%%%%%%%%%%%%%%%%%%%%%%%%%%%%%%%%%%%%%%%%%%%%%%%%%%%%%

\title{
    这里写标题。标题将会以较大字号居中打印
}

\author{
    张三~~~~李四~~~~王小二$^1$
 \\ \\
$^1$这个是小组长
}
% tex 文字编排比较自由散漫,任意个连续空格等效于一个空格。如要有意拉开
% 空格,可像上面那样用波浪号 '~' 填充。
% 任意个连续空行等效于一个空行。空行是自然段的分界
% title 和 author 环境里不允许有多余的空行。如果有意要制造行距,两个反斜线
% 表示增加一个空行

\date{\today}
% 今天的日期。也可以强制写成 \date{2019年2月29日}

% 以上称作'导言区(preamble)'。从 \begin{document}开始是正文

\begin{document}
\maketitle     % \maketitle 将生成标题、作者和日期

\begin{abstract}
在abstract环境里写摘要。摘要要求简单明了,让读者通过短短几句话了解文章的
大概内容。

本文试图用最小案例介绍使用\LaTeX 进行排版的基本方法。为避免初学者的恐惧,
本文采用了最简单的做法,仅作为入门向导。一些做法可能不是最合理的,建议
在学习过程中参考更多的文档\cite{tex}。

中文请使用utf--8编码。体面点的文档编辑器都知道什么是utf--8,在执行
``另存为...''操作的时候会出现编码格式的选项。
  
\end{abstract}

\textbf{关键词:不要~~~~超过~~~~三个词}
% 如果需要强调,可通过 \textbf 命令将字体加粗。

\section{这是一节}

对于习惯word工具的人来说,使用\LaTeX 最大的问题在于转变思想。word不是
专门用于排版的工具(大多数人使用word时只是以为自己在排版)。而使用
\LaTeX 时,作者只需要关注文章的内容和逻辑结构,不要过多地考虑打印版是
什么样子。例如,你应该考虑哪些文字组成一个自然段,哪些自然段构成一个章节。
你无需考虑一段文字在什么地方换行,这一小节的编号是几.几。

\subsection{章节}
article 模板的层次结构由 part、section、subsection、subsubsection、
paragraph、subparagraph 六层构成。一般小篇幅的文章不使用 part 而是从
section 开始编起。

\subsubsection{这是某一节里的小节}
不要手工给文章的逻辑层次编号。使用结构命令自动编号可保证风格的统一,改变
风格也很容易,只需要改变模板一处而不用逐一改变tex文档。

\subsection{插图}
如果要插图,使用 figure 环境。见图\ref{njulogo}

\section{图表的编号和引用}
让 tex 给图、表自动编号。 图表属于浮动对象,排版后,他们的位置可能会发生
变化。因此,不要用``如\textbf{下图}所示''、``见\textbf{上表}''这样的文字。
\subsection{插图比较简单}
图、表的编号和引用由 $\backslash$ref\{ \} 和$\backslash$label\{ \}
一对命令构成。

图表的名称用 $\backslash$caption 构造。一般图的 caption 在图的下方,
而表格的 caption 在表的上方。为美观起见,不与文字混排的图表通常都要居中。

\begin{figure}
\centering   % 对齐命令: \flushleft(缺省)、\flushright、\centering
\includegraphics[width=3cm]{njulogo.pdf}
% width、height 可以用来控制图的大小,rotate控制旋转角度
\caption{这是南大的logo} \label{njulogo}
\end{figure}

\subsection{表格稍微复杂一些}
表格的环境是 tabular,通常还要在外面再套一层 table 环境,使之浮动。
如果实在想让浮动对象出现在固定位置,可以试着给浮动环境加一个选项``!h'',
见表\ref{pub_tools}(下表)。

\begin{table}[!h]
\centering
\caption{几种排版工具比较}  \label{pub_tools}
\begin{tabular}{c|c|c}
    \hline
  排版工具 & 特点 & 使用权   \\ \hline
   word    & 所见所得 & 咨询MS \\
    方正    & 专业排版工具 &   \\
   LaTeX   &  ``所见非所得'' & 开源、免费\\ \hline
\end{tabular}
\end{table}

同样,数学环境也应自动编号,比如公式(\ref{form1})。

\begin{equation} \label{form1}
    \int \sin(x) dx = -\cos(x) + C
\end{equation}

\subsection{图的格式}
 xelatex 直接支持 png、jpg、bmp、pdf 图形格式,其中 pdf 是矢量图,如果有
 这种格式,尽量使用它。
 
\section{参考文献的引用}

参考文献使用 thebibliography 环境\cite{luo}。每一篇文献是一个带标签的
$\backslash$bibitem 项。标签供 $\backslash$cite 命令引用。可适当改变作者、
书名或文章标题、出版社等几部分的字体。同样,参考文献顺序也不要手工编号。

\section{文档生成}

Linux系统中通过编译命令 xelatex 将 .tex 格式的文件生成 pdf。如果有交叉引用,
可能需要编译两次(更复杂的文档可能要编译三次)。

Windows系统一般会有集成编辑环境,通过菜单项操作生成 pdf 文档。

\begin{thebibliography}{99}
\bibitem{wu} \textit{吴承恩}. ~~ \textbf{西游记},~~ 人民文学出版社,1995.10
\bibitem{tex} \textit{Tobias Oetiker}.~~
    \textbf{A not so short introduction to \LaTeX-2$\epsilon$},
    原版 Version4.20, May 31, 2006
\bibitem{luo} \textit{罗贯中}. ~~ \textbf{三国演义},~~ 大众书局,2000.1
\end{thebibliography}
\newpage

\lstinputlisting[language={[LaTeX]TeX},caption={源文本清单}]{template.tex}

\end{document}
