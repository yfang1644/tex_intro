\documentclass[10pt,t]{beamer}
\usepackage[fontset=ubuntu]{ctex}
\usepackage{graphicx}
\usepackage{listings}
\usepackage{ulem}
\usepackage{boxedminipage}
\input{style.tex}

\lecture[1]{软件开发规范}{}

\begin{document}
\begin{frame}
  \maketitle
\end{frame}

\begin{frame}\frametitle<presentation>{Contents}
  \tableofcontents

\end{frame}

\section{C语言代码风格}
\begin{frame}{格式和缩进}
\begin{itemize}
    \item 代码块缩进
    \item 逻辑块之间空行
    \item 长语句断行

    \item more coding style ref. Kernel \alert{Documentation/CodingStyle}.
\end{itemize}
\end{frame}

\begin{frame}{注释}
\begin{itemize}
    \item 源文件的顶部注释: 文件名, 作者, 版本历史等
    \item 函数注释: 功能, 参数, 返回值, 错误码等
    \item 语句组上方注释
    \item 代码行右侧注释
    \item 复杂的结构体定义注释
    \item 复杂的宏定义和变量声明注释
\end{itemize}
\end{frame}

\begin{frame}{变量命名}
命名要清晰明了,易于理解
\begin{itemize}
    \item 使用完整单词、去元音缩写、首字母缩写...

        \alert{count} $\to$ \alert{cnt}, \alert{block} $\to$ \alert{blk},
        \alert{length} $\to$ \alert{len},
        \alert{internationalization} $\to$ \alert{i18n} ...
    \item 变量用下划线分割的小写字母单词, 常量用大写字母单词加下划线

        \alert{radix\_tree\_insert}, \alert{RADIX\_TREE\_MAP\_SHIFT}
    \item 全局变量和函数命名要力求详细, 宁可多用几个单词多写几个下划线

    \item 局部变量和内部函数命名可以略短,但不能太简单.
        循环变量 \alert{i}, \alert{j}, \alert{k} 等除外
    \item 不建议用汉语拼音命名标识符

\end{itemize}
    其他命名法:驼峰命名法、匈牙利命名法...
\end{frame}

\begin{frame}{函数设计原则}
\begin{itemize}

    \item 函数功能尽可能单一, 不求面面俱到 (考虑重用性和维护性)
    \item 函数内缩进层次不宜过多, 一般不超过 3 层。否则应考虑分割
    \item 函数行数一般不超过 2 屏 (48 行), 否则应考虑分割
    \item 函数名称通常应包含动词, 如
        
        \alert{get\_current()}、\alert{radix\_tree\_insert()}

    \item 重要函数上方必须加注释, 说明功能、参数、返回值、错误码等

    \item 函数内局部变量数不宜超过 10 个, 否则应考虑分割
\end{itemize}
\end{frame}

\begin{frame}{调试指令}
\begin{itemize}
    \item printf() 输出
    \item assert() 断言方法
\end{itemize}

\end{frame}


\section{软件质量}
\begin{frame}{软件质量的功能性指标}
\begin{itemize}
    \item 正确性 (Correctness)

        按照需求正确执行任务的能力
    \item 健壮性 (Robustness)
    \begin{itemize}
        \item 容错能力
        \item 恢复能力
    \end{itemize}
    \item 可靠性 (Reliability)

        平均无故障时间 (MTTF)
\end{itemize}
\end{frame}

\begin{frame}{软件质量的非功能性指标}
\begin{itemize}
    \item 性能 (Performance) $\to$ 算法复杂度分析
    \item 易用性 (Usability) (由用户判断)
    \item 清晰性 (Clarity) ,或称可读性 (Readability)
    \item 安全性 (Security) $\to$ 防止被非法入侵
    \item 可扩展性 (Extendibility) $\to$ 规模和复杂性
    \item 兼容性 (Compatibility)
    \item 可移植性 (Portability)
    \begin{itemize}
        \item 设备相关程序和设备无关程序分离
        \item 功能模块和用户界面分离
    \end{itemize}
\end{itemize}
\end{frame}

\end{document}
