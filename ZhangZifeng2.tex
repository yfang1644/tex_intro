%%============================================================================
%%
%%       Filename:  table2.tex
%%
%%    Description:  thesis review
%%
%%        Version:  1.0
%%        Created:  09/08/2018 09:23:46 PM
%%       Revision:  none
%%
%%         Author:  Fang Yuan (yfang@nju.edu.cn)
%%   Organization:  nju
%%      Copyright:  Copyright (c) 2018, Fang Yuan
%%
%%          Notes:  
%%                
%%============================================================================
\documentclass[a4paper,12pt]{article}
\usepackage[fontset=none]{ctex}
\usepackage[paperwidth=15cm, paperheight=23.5cm,
            margin=0pt, top=3.5cm, bottom=1.5ex,
            left=0pt,right=0pt]{geometry}
\usepackage[a4,frame,center]{crop}
\usepackage{ltxtable}

\setCJKfamilyfont{caiyun}{STCaiyun}
\setCJKfamilyfont{songti}{HYZhongSongS}
\setCJKfamilyfont{kaiti}{AR PL UKai CN}
\setCJKfamilyfont{fangsong}{STFangsong}
\setCJKfamilyfont{heiti}{WenQuanYi Micro Hei}
\setCJKfamilyfont{xinwei}{STXinwei}
\setCJKfamilyfont{lishu}{LiSu}
\setCJKmainfont{SimSun}
%\setCJKmainfont[BoldFont={WenQuanYi Micro Hei}, ItalicFont={STFangsong}]{SimSun}
\setCJKsansfont{WenQuanYi Micro Hei}
\setCJKmonofont{AR PL UKai CN}

%\setmainfont{Times New Roman}
\setsansfont{Droid Sans}
\setmonofont{DejaVu Sans Mono}

\newcommand*{\songti}{\CJKfamily{songti}}   % 宋体
\newcommand*{\kaishu}{\CJKfamily{kaiti}}     % 楷书
\newcommand*{\fs}{\CJKfamily{fangsong}}   % 仿宋
\newcommand*{\heiti}{\CJKfamily{heiti}}     % 黑体
\newcommand*{\wei}{\CJKfamily{xinwei}}    % 新魏
\newcommand*{\lishu}{\CJKfamily{lishu}}   % 隶书
\newcommand*{\cy}{\CJKfamily{caiyun}}     % 彩云

\newcommand*\markC{%
\begin{picture}(0,0)
    \linethickness{1.3pt}\unitlength 1cm
    \put(0,0){\framebox(15,20)[c]{}}
\end{picture}}
        
\newcommand*\markinfo{
    \shortstack{
    {\bf \Large 南~京~大~学\vspace{5mm}}
        \\
    {\renewcommand{\CJKglue}{\hskip .4em}
       \bf \Large 研究生学位申请书$\cdot$附表二\vspace{5mm}}
        \\
    \fontsize{16}{16}\selectfont
    \fs \large 学位(毕业)论文评阅意见表
    }
}
\cropdef[\markinfo]\relax\relax\markC\relax{istin}
\crop[istin]
\addtolength{\leftskip}{1em}
\addtolength{\rightskip}{1em}
    
\begin{document}
\pagestyle{empty}
\fontsize{12}{16}\selectfont
\begin{center}
\begin{tabularx}{\textwidth}{c|c|c|c|c|X|X}
研究生姓名 & \multicolumn{2}{c|}{张梓峰} & 入学时间 & 2016 & 攻读学位
        & 工程硕士\\\hline
所~学~专~业   & \multicolumn{2}{c|}{电子与通信工程}
        & 研究方向 & \multicolumn{3}{l|}{音频信号处理}\\\hline
论~文~题~目 & \multicolumn{6}{l|}{基于音频信号的高带宽数据传输系统的设计与实现}\\\hline
    导师姓名 & 方元 & 职称 & 副教授 & 所在单位 &
    \multicolumn{2}{l|}{南京大学电子学院}\\\hline
    评阅人姓名  &  & 职称 &      & 所在单位 & 
    \multicolumn{2}{l|}{南京大学电子学院}\\\hline
\end{tabularx}
\vskip 8mm
\Large \kaishu 学术评语\end{center}

\fontsize{12}{14}\selectfont

随着汽车工业的发展, 人们对车载电子系统的要求越来越高。
而当电子系统的软件需要更新时, 又时常缺少廉价便捷的方法。
这时人们想到了音响系统, 它除了可以传输模拟信号以外, 也可以设法让它传输
数字信号。张梓峰同学的论文《基于音频信号的高带宽数据传输系统的设计与实现》
即是这一设想的有益尝试。

本文首先分析了基于模拟音频信号传输数据的背景及现状, 介绍了各项技术的优缺点;
接着讨论了本设计所需的理论和核心技术, 包括音频合成、频谱分析技术、窗函数, 
以及相应的校验和签名技术等; 根据传输系统的具体需求, 设计了数据传输系统模型,
并设计实现了此传输系统。其中根据具体情况, 设计了一套传输协议。

论文详细介绍了各模块的具体设计过程。最后在PC上使用 Python 语言对系统
进行了模拟, 证明该方案的有效性, 单通道传输性能可达14.3kbps。
该方案可被移植到相应的项目里, 以实现软件更新升级的功能。

论文选题符合专业培养目标, 具有实用价值。作者在选题和论文撰写过程中,
查阅了大量相关文献, 结合工作经验, 灵活运用已学知识解决课题中遇到的问题。
论文表现出作者具有扎实的理论基础, 具备独立从事科学研究的能力, 工作细致踏实, 具有团结协作精神。

论文达到硕士学位论文水平, 同意该生进行硕士论文答辩和硕士学位申请。

\vfill
\hfill {\heiti 评阅人} \underline{\hskip 35mm} (签章)\qquad\qquad
\vskip 3mm
\hfill 年\hskip 12mm 月\hskip 12mm 日\qquad\qquad

\noindent(可加附页)

\end{document}
