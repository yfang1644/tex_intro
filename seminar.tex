\documentclass[10pt]{beamer}
\usepackage[fontset=none]{ctex}
\usepackage{graphicx}
\usepackage{listings}
\usepackage{ulem}
\input{style.tex}

\lecture[1]{实验教学和实验室建设}{}

\setCJKmainfont[BoldFont={WenQuanYi Micro Hei},
        ItalicFont={STFangsong}]{AR PL UMing CN}
\setCJKsansfont[BoldFont={WenQuanYi Micro Hei},
        ItalicFont={AR PL UKai CN}]{HYZhongSongS}
\setCJKfamilyfont{kaiti}{AR PL UKai CN}

\begin{document}
\begin{frame}
  \maketitle
\end{frame}

\section{课程设计}
\begin{frame}[t]{实验课程设计}
    传统上,把实验类型分成
\begin{itemize}
    \item \alert{验证性实验}
    \item \alert{设计(综合)实验}
    \item \alert{研究型(探索型)实验}(or whatever)
    \item 。。。
\end{itemize}
    对于给定目标的实验,充其量不超过设计实验的高度
    (所有人都能做出来,或者所有人都做不出来),不应以难度拔高地位。

    ~\\

    基础实验必须要有验证性实验,不宜过份强调研究型实验。
    (尤其是用综合实验冒充研究型实验)

    ~\\

    重要的是,通过实验课打算培养什么?(知识体系/科学精神)
    ~\\ \ \\ \ \\ \ \\ {\tiny 国外课程设计}
\end{frame}

\section{实验教学}
\begin{frame}[t]{过程控制}
    experiment  (peri = to go through), not test.

    ~\\

    过程比结果更重要
   
    ~\\

    动脑比动手更重要

    ~\\

    对待学生的态度
\begin{itemize}
    \item 守信---言出必行,做不到的别说出来
    \item 耐心---我们以为他们应该知道的问题,他们未必知道 (原因比较复杂)
    \item 尊重---尽量不当众点名批评,...
\end{itemize}

\end{frame}


\begin{frame}[t]{实验报告}
\begin{itemize}
    \item 为什么要做实验 (缺一堂课)
    \item 为什么要写实验报告
    \item 如何写实验报告 (缺一堂课)---如何引导学生写好实验报告
    \item 实验报告涉及到数据整理、过程分析、思考、创新。。。
    \item 关于电子实验报告的想法 (电子版 vs. 电子素材)
\end{itemize}

典型问题:罗列数据,照本宣科、分析乏力,不下结论
    (格式就不说了)
\end{frame}

\section{教师发展}

\begin{frame}[t]{教师的发展空间}
实验中心研究层次低,但仍有发展空间
\begin{itemize}
    \item 参与课题组项目
    \item 专利
    \item 实验研究论文
    \item 教材 (比如模拟电路实验教材。。。)
    \item 参与各种组织 (包括竞赛命题)
\end{itemize}
\end{frame}


\begin{frame}[t]{人际氛围}
\end{frame}
\end{document}
