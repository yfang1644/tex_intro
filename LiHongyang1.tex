%%============================================================================
%%
%%       Filename:  table1.tex
%%
%%    Description:  thesis review
%%
%%        Version:  1.0
%%        Created:  09/08/2018 09:23:46 PM
%%       Revision:  none
%%
%%         Author:  Fang Yuan (yfang@nju.edu.cn)
%%   Organization:  nju
%%      Copyright:  Copyright (c) 2018, Fang Yuan
%%
%%          Notes:  
%%                
%%============================================================================
\documentclass[a4paper,12pt]{article}
\usepackage[fontset=none]{ctex}
\usepackage[paperwidth=15cm, paperheight=23.5cm,
            margin=0pt, top=3.5cm, bottom=1.5ex,
            left=0pt,right=0pt]{geometry}
\usepackage[a4,frame,center]{crop}
\usepackage{ltxtable}

\setCJKfamilyfont{caiyun}{STCaiyun}
\setCJKfamilyfont{songti}{HYZhongSongS}
\setCJKfamilyfont{kaiti}{AR PL UKai CN}
\setCJKfamilyfont{fangsong}{STFangsong}
\setCJKfamilyfont{heiti}{WenQuanYi Micro Hei}
\setCJKfamilyfont{xinwei}{STXinwei}
\setCJKfamilyfont{lishu}{LiSu}
\setCJKmainfont{SimSun}
%\setCJKmainfont[BoldFont={WenQuanYi Micro Hei}, ItalicFont={STFangsong}]{SimSun}
\setCJKsansfont{WenQuanYi Micro Hei}
\setCJKmonofont{AR PL UKai CN}

%\setmainfont{Times New Roman}
\setsansfont{Droid Sans}
\setmonofont{DejaVu Sans Mono}

\newcommand*{\songti}{\CJKfamily{songti}}   % 宋体
\newcommand*{\kaishu}{\CJKfamily{kaiti}}     % 楷书
\newcommand*{\fs}{\CJKfamily{fangsong}}   % 仿宋
\newcommand*{\heiti}{\CJKfamily{heiti}}     % 黑体
\newcommand*{\wei}{\CJKfamily{xinwei}}    % 新魏
\newcommand*{\lishu}{\CJKfamily{lishu}}   % 隶书
\newcommand*{\cy}{\CJKfamily{caiyun}}     % 彩云

\newcommand*\markC{%
\begin{picture}(0,0)
    \linethickness{1.3pt}\unitlength 1cm
    \put(0,0){\framebox(15,20)[c]{}}
\end{picture}}
        
\newcommand*\markinfo{
    \shortstack{
    {\renewcommand{\CJKglue}{\hskip 1em} \bf \Large 南京大学\vspace{5mm}}
        \\
    {\renewcommand{\CJKglue}{\hskip .4em}
       \bf \Large 研究生学位申请书$\cdot$附表一\vspace{5mm}}
        \\
    \fontsize{16}{16}\selectfont
    \fs \large 指导教师对研究生学位(毕业)论文评语
    }
}
\cropdef[\markinfo]\relax\relax\markC\relax{istin}
\crop[istin]
\addtolength{\leftskip}{1em}
\addtolength{\rightskip}{1em}
    
\begin{document}
\pagestyle{empty}
\fontsize{12}{16}\selectfont
\begin{center}
\begin{tabularx}{\textwidth}{c|c|c|c|c|X|X}
研究生姓名 & \multicolumn{2}{c|}{李洪洋} & 入学时间 & 2014.03 & 攻读学位
        & 工程硕士\\\hline
所~学~专~业 & \multicolumn{2}{c|}{电子与通信工程}
        & 研究方向 & \multicolumn{3}{l}{电路与系统}\\\hline
论~文~题~目 & \multicolumn{6}{l}{125k低频触发+2.4G 11n传输RFID产品设计}\\\hline
\end{tabularx}
\vskip 8mm
\Large \kaishu 导师评语\end{center}

\fontsize{12}{14}\selectfont

随着社会服务业发展,对人员及物品定位的需求越来越多。
蓝牙、RFID、红外、ZIGBEE 等定位技术也在不断地得到应用,
但作为室内定位,它们都受到一定的限制。
以WiFi定位标签为代表的被动定位技术的应用给企业和人们的生活带来更多的
积极变化。
论文以WiFi标签定位技术为研究对象,具有一定的理论研究意义,并具有较好的
实用价值。

论文首先介绍了人员及物品定位的应用背景,展现了被动定位的市场化机遇,
同时也指出了WiFi定位标签技术的技术劣势:精度和供电问题。
论文针对通常WiFi定位应用存在的一些问题,结合RFID技术,设计了125k低频触发器、
有源2.4G 11n传输的电子标签、以及标签管理器,在使用原有的2.4G 11n AP/定位
基站基础上,组成125k低频触发+2.4G 11n传输RFID产品系统,以提高WiFi定位精度
并延长供电时长。论文详细讨论了系统的设计依据,完成了相应的硬件和软件,
最终实现系统联调,达到了预期的设计目标,并实现了产品的市场化。

论文选题符合专业培养目标,能够达到综合训练目标,时效性较强。作者在
选题和论文撰写过程中,查阅了大量相关文献,结合工作经验,灵活运用已学
知识解决课题中遇到的问题。论文表现出作者具有扎实的理论基础,具备独立
从事科学研究的能力,工作细致踏实,具有团结协作精神。

本人认为论文达到硕士学位论文水平,同意该生进行硕士研究生论文答辩和
硕士学位申请。

\vfill
\hfill {\heiti 指导教师} \underline{\hskip 35mm} (签章)\qquad\qquad
\vskip 5mm
\hfill 年\hskip 12mm 月\hskip 12mm 日\qquad\qquad

\noindent(可加附页)

\end{document}
